\documentclass[t]{beamer}

\usepackage[utf8]{inputenc}
\usepackage[T1]{fontenc}

%%%%%%%%%%%%%%%%%%%%%%%%%%%%%%%%%%%%%%%%%%%%%%%%%%%%%%%%%%%%%%%%%%%%%%%%%%%%%%%%%%%%%%%%%%%%%%%%%%%%
%% STYLE
%%%%%%%%%%%%%%%%%%%%%%%%%%%%%%%%%%%%%%%%%%%%%%%%%%%%%%%%%%%%%%%%%%%%%%%%%%%%%%%%%%%%%%%%%%%%%%%%%%%%

% the possible options for the 'i2beamer' package are:
% - 'english' (if the slides are in english) 
% - 'wide'    (if the slides should use a 16:9 aspect ratio)
% - 'logo'    (if the logo of the Programming Systems Group should be included)
% - 'plain'   (if the background image of the title page should be omitted)
\usepackage[wide]{i2beamer}

%%%%%%%%%%%%%%%%%%%%%%%%%%%%%%%%%%%%%%%%%%%%%%%%%%%%%%%%%%%%%%%%%%%%%%%%%%%%%%%%%%%%%%%%%%%%%%%%%%%%
%% PACKAGES
%%%%%%%%%%%%%%%%%%%%%%%%%%%%%%%%%%%%%%%%%%%%%%%%%%%%%%%%%%%%%%%%%%%%%%%%%%%%%%%%%%%%%%%%%%%%%%%%%%%%

\usepackage{lipsum} % just for testing

%%%%%%%%%%%%%%%%%%%%%%%%%%%%%%%%%%%%%%%%%%%%%%%%%%%%%%%%%%%%%%%%%%%%%%%%%%%%%%%%%%%%%%%%%%%%%%%%%%%%
%% INFO
%%%%%%%%%%%%%%%%%%%%%%%%%%%%%%%%%%%%%%%%%%%%%%%%%%%%%%%%%%%%%%%%%%%%%%%%%%%%%%%%%%%%%%%%%%%%%%%%%%%%

\title{ProgLLM}
\subtitle{Status Report}

\author{Lisa Rebecca Schmidt}
\institute[\PSName{}]{Friedrich-Alexander-Universität Erlangen-Nürnberg}

\date[Datum]{9. Feb. 2024}

%%%%%%%%%%%%%%%%%%%%%%%%%%%%%%%%%%%%%%%%%%%%%%%%%%%%%%%%%%%%%%%%%%%%%%%%%%%%%%%%%%%%%%%%%%%%%%%%%%%%
%% DOCUMENT
%%%%%%%%%%%%%%%%%%%%%%%%%%%%%%%%%%%%%%%%%%%%%%%%%%%%%%%%%%%%%%%%%%%%%%%%%%%%%%%%%%%%%%%%%%%%%%%%%%%%

\begin{document}

\begin{frame}[plain,c]
  \titlepage
\end{frame}


% \begin{frame}
%   \frametitle{Agenda}

%   \tableofcontents
% \end{frame}


\begin{frame}
  \frametitle{Problems Encountered}
  \framesubtitle{ }

  \begin{itemize}
    \item Obsolete libraries/APIs/dependencies
    \begin{itemize}
        \item remedy: ask for the version!
        \item very annoying: it even had old versions of the openai apis...
    \end{itemize}
    \item "This task would require a team of developers ..." / "This task is highly complex ..." / "I can provide a basic implementation which needs to be extended"
    \begin{itemize}
        \item remedy: Prompt: "You are a team of experienced [XYZ-] engineers".
    \end{itemize}
    \item Systems design can be overwhelming, as it will present many different technologies which you may not yet know.
    \item Configurations are always the same! Especially when you design microservices, it will always use the same ports.
    \item The interfaces which are built are not actually used! -> you have to know how apis/sockets/etc work
  \end{itemize}


\end{frame}

\subsection{Fließtext}

\begin{frame}
  \frametitle{Insights}
  \begin{itemize}
    \item It still makes sense to learn new technologies
    \begin{itemize}
        \item If you already know the technologies, it will speed you up even more.
        \item You will have a much easier time debugging.
        \item You know 'best' practices.
        \item You know what you are doing ;)
        \item I failed miserably when I trusted GPT-4 to use React (I don't know React).
    \end{itemize}
    \item LLMS can speed up coding significantly.
    \item The LLM can actually handle simple systems-architecture problems.
    \item You are very much forced to build decoupled and modular code (which is good) because the LLM will have "forgotten" your specs.
    \begin{itemize}
        \item Bonus: Let the LLM generate interface specs, these you can then put in the prompt if you need a specific module.
    \end{itemize}
    \item You are also very much forced to use Infrastructure-as-Code and templates, because LLMS can only generate text.
    \item You will lose a lot of time if you switch back and forth between Google/Stack Overflow and your LLM - stick to one!
\end{itemize}

  
\end{frame}



\end{document}
